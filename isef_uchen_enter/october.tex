%!TEX TS-program = xelatex
% see https://www.writelatex.com/coursera/latex/5.2.2

\documentclass[a4paper,11pt]{extarticle}

% Russian fonts
\usepackage[english,russian]{babel}
\usepackage[utf8]{inputenc}
\usepackage{fontspec}
\defaultfontfeatures{Ligatures={TeX},Renderer=Basic}
\setmainfont[Ligatures={TeX,Historic}]{Times New Roman}
\setmonofont{Courier New}
\usepackage{indentfirst}
\usepackage{parskip}
\usepackage{tikz}
\usepackage{hyperref}
\usepackage[nameinlink]{cleveref}
\usepackage{booktabs}
\usepackage{tabularx}
\usepackage{amssymb}
\usepackage{wrapfig}
\usepackage[bottom]{footmisc}
\linespread{1}
\hypersetup{final=true}
\addto\captionsngerman{
	% Second argument is singular, third is plural
	\crefname{figure}{abb.}{abb.}
	\Crefname{figure}{Abb.}{Abb.}
}
\crefname{figure}{рис.}{рис.}
\frenchspacing

% Margins
% \usepackage[top=20mm, bottom=20mm, left=25mm, right=10mm]{geometry} % GOST
\usepackage[top=20mm, bottom=20mm, left=20mm, right=20mm]{geometry}

% Paragraph spacing
\setlength{\parindent}{0em}
\setlength{\parskip}{1em}
\usepackage{setspace}
\usepackage{graphicx}
\usepackage{tabulary}

\setcounter{secnumdepth}{1}
\title{\vspace{-2.5cm}Анкета Захарова Ильи, школа №1533 ЛИТ\vspace{-1cm}}
\date{}
\begin{document} % конец преамбулы, начало документа
\maketitle
\section{Результаты в октябре}

	\subsection{Изучение Real Mode ассемблера (16bit)}
\begin{itemize}
	\item Загрузочный сектор: написание своего загрузочного кода
	\item Написание клеточного автомата в загрузочном секторе
	\item BIOS - прерывания: int 13h, int 15h
	\item Работа с VideoBIOS: работа с int 10h
	\item Начало работы с эмуляторами Bochs, QEMU
\end{itemize}
\subsection{Изучение Protected Mode (32bit)}
\begin{itemize}
	\item Управление памятью: глобальная таблица дескрипторов (GDT)
	\item Изучение Big Real Mode
	\item Изучение структуры исполняемого файла ELF
	\item Изучение компоновки кода на Assembler и кода на языке C
	\item Начало разработки в Protected Mode на языке C
	\item Чтение карты доступной физической памяти
	\item Виртуальная память: настройка страничного отображения (Paging)
	\item Обработка прерываний: таблица прерываний в защ. режиме (IDT)
	\item Перенос кода ядра и таблиц выше 1 МБ
	\item Изучение механизмов аппаратной многозадачности. Структура TSS
\end{itemize}
\subsection{Изучение топологии ядер и процессоров (ACPI: SRAT, SLIT)}
\begin{itemize}
	\item Изучение APIC, XAPIC (LAPIC, IOAPIC, ACPI: MADT)
\end{itemize}
\subsection{Изучение Long Mode (64 bit, x86-64)}
\begin{itemize}
	\item Изучение особенностей TSS и IDT в 64-битном режиме
	\item Изучение особенностей страничной адресации памяти (PAE)
\end{itemize}
\subsection{Изучение аппартных механизмов виртуализации VMX (VT-i, VT-d)}
\begin{itemize}
	\item Подготовка и включение VMX-режима
	\item Подготовка управляющих структур для виртуальной машины
	\item Создание обработчика VMCall
	\item Создание обработчиков событий
\end{itemize}
\subsection{Другое}
\begin{itemize}
	\item Сборка проекта с помощью утилиты \texttt{make}
	\item Портирование проекта с собственного загрузчика на GRUB
	\item Настройка VGA, VBE как выводов терминала системы
\end{itemize}
\section{После октября}
	Минус означает ещё не решённую задачу.
\begin{itemize}
	\item Управление памятью
	\renewcommand{\labelitemii}{$\circ$}
	\begin{itemize}
		\item Высокоуровневый доступ к структурам paging-а
		\item Выделение физической памяти (binary buddy allocator)
		\item Куча (heap)
	\end{itemize}
	\item Доступ к отладочной консоли через Intel LPSS UART\footnote{определяется в DBGP, PCI номер устройства: вендор 8086h, id устройства A166h}
	\item Обработка ACPI таблиц: MADT, DBGP
	\item Инициализация многоядерности: отправление SIPI, init-IPI
	\item Разработка планировщика для задач на BSP (round-robin)
	\item Написание документации проекта на Doxygen [in process]
	\item[$-$] Создание расширенных таблиц страничной трансляции (EPT)
	\item[$-$] Создание BIOS для виртуальных машин
\end{itemize}\par
	\textbf{Все практические задачи были решены собственными силами автора. Коррективы в исходный код вносили научный руководитель и консультанты.}
\end{document} % конец документа