%!TEX TS-program = xelatex
% see https://www.writelatex.com/coursera/latex/5.2.2

\documentclass[a4paper,12pt]{extarticle}

% Russian fonts
\usepackage[english,russian]{babel}
\usepackage[utf8]{inputenc}
\usepackage{fontspec}
\defaultfontfeatures{Ligatures={TeX},Renderer=Basic}
\setmainfont[Ligatures={TeX,Historic}]{Times New Roman}
\setmonofont{Courier New}
\usepackage{indentfirst}
\usepackage{parskip}
\usepackage{tikz}
\usepackage{hyperref}
\usepackage[nameinlink]{cleveref}
\usepackage{booktabs}
\usepackage{tabularx}
\usepackage{amssymb}
\usepackage{wrapfig}
\usepackage[bottom]{footmisc}
\linespread{1.5}
\hypersetup{final=true}
\addto\captionsngerman{
	% Second argument is singular, third is plural
	\crefname{figure}{abb.}{abb.}
	\Crefname{figure}{Abb.}{Abb.}
}
\crefname{figure}{рис.}{рис.}
\frenchspacing

% Margins
% \usepackage[top=20mm, bottom=20mm, left=25mm, right=10mm]{geometry} % GOST
\usepackage[top=20mm, bottom=20mm, left=20mm, right=20mm]{geometry}

% Paragraph spacing
\setlength{\parindent}{0em}
\setlength{\parskip}{1.5em}
\usepackage{setspace}
\usepackage{graphicx}
\usepackage{tabulary}

\setcounter{secnumdepth}{0}

\begin{document} % конец преамбулы, начало документа
	Секция: Программирование\\
	Школа №1533 "ЛИТ"\\
	119296 г.Москва, Ломоносовский проспект, дом 16\\
	тел. (499) 133-24-35, e-mail: info@lit.msu.ru\par
	{\Large \textbf{Разработка гипервизора Jinet}}\par
	Захаров Илья\\
	класс: 11\\
	123060 г.Москва. ул.Маршала Бирюзова, д.20к2, кв.54\\
	тел.: 8 (916) 620-86-51, e-mail: ilya101010@gmail.com\par
	Научный руководитель: Байков Борис Камалевич,\\ ведущий программист, руководитель группы, <<Т-Платформы>>
	\\\par
	
	\setlength{\parskip}{0.5em}
	\textbf{Цель работы} -- это создание минимального монитора виртуальных машин (гипервизора) с использованием механизмов аппаратной виртуализации архитектуры x86-64 (AMD64). \\\par
	С каждым годом технологии виртуализации всё глубже и глубже входят в мир информационных технологий, находя применения в самых разных областях IT:
	\begin{enumerate}
		\item изоляция серверных систем для обеспечения их безопасности
		\item эффективное сегментирование ресурсов компьютера
		\item одновременное использование разных ОС на настольном компьютере
		\item отладка гостевых систем 
	\end{enumerate}
	\par Большиство из ныне существующих гипервизоров массивны и поддерживают множество функций. Реализованный в рамках этой работы гипервизор может послужить базой для гипервизоров, заточенных под конкретные задачи. Мотивацией для работы стали интерес и актуальность технологий виртуализации и прямого программирования оборудования. Исследованы механизмы виртуализации, получен опыт разработки кода управления ими. \par
	В ходе работы над проектом было изучено большое количество документации по процессорам Intel (\cite{intel}) и AMD (\cite{amd}), документация \texttt{gnu make} (\cite{make}), \texttt{gnu ld} (\cite{ld}), \texttt{gcc} (\cite{gcc}), \texttt{fasm} (\cite{fasm}), формата ELF (\cite{elf}), таблиц ACPI (\cite{acpi}).\par
	Гипервизор Jinet был написан на языках программирования ассемблер (диалекты \texttt{fasm} и \texttt{as}) и C (компилятор \texttt{gcc}). Сборка проекта осуществляется с помощью сборщика \texttt{gnu ld} и утилиты \texttt{gnu make}. В качестве системы контроля версий используется \texttt{git} и хостинг GitHub. Гипервизор позволяет запускать код в изолируемом окружении виртуальной машины. Код, инициализирующий подсистемы компьютера и режим VMX, можно использовать для демонстрации возможностей виртуализации, а также в учебных целях.\par
	Исходный код проекта находится под лицензией MIT.
	\section{Литература}
	\nocite{*}
	\renewcommand{\section}[2]{}%
	\bibliographystyle{../utf8gost705u}
	\bibliography{../biblio}
\end{document} % конец документа

