\documentclass[final]{beamer}
\mode<presentation>
{
	\usetheme{Berlin} % тема
}
\usepackage{times}
\usepackage{amsmath,amssymb}
\usepackage[utf8]{inputenc}
\usepackage[russian]{babel}
\usepackage[orientation=portrait,size=a0,scale=1.1]{beamerposter}
% ориентация может быть landscape.
% размер - другой, какой захотите, например, a1,
% или какой-нибудь свой, size=custom,width=200,height=120
% scale - множитель на размер шрифта.  Очень удобно для подгонки
% содержимого постера для идеального заполнения пространства!
\usepackage{multicol}  % в несколько колонок

% нумерация рисунков в презентации
\setbeamertemplate{caption}[numbered]

\newcommand{\celcius}{\,^{\circ}\mathrm{C}}
\graphicspath{{pics/}}
\listfiles

% Display a grid to help align images
%\beamertemplategridbackground[1cm]

\title{\Huge Phase splitting with temperature in perovskite-type
	strontium cobaltite \\doped with Fe and Nb}

\author{Maxim G. Ivanov\inst{1}, Alexander N. Shmakov\inst{1,2},
	Sergey V. Tsybulya\inst{1,2}, Olga Yu. Podyacheva\inst{1}}
\institute[] % (optional, but mostly needed)
{
	\inst{1}%
	Boreskov Institute of Catalysis SB RAS, Russia
	\\
	\inst{2}%
	Novosibirsk State University, Russia
}

\date[May. 27-30th, 2009]{May. 27-30th, 2009}

\begin{document}
	\begin{frame}{}
	\vspace{-1cm}
	\begin{columns}[t]  %выравнивание колонок постера по верху
		%%%%%%%%%%%%%%%%%%%%%%%%%%%%%%%%%%%%%%%%%%%%%%%%%%%%%%%%%%%%%%%%%%%%%%%%%%%%%%%%%%%%%%%%%%%%%%%%%%%%
		%%%%%%%%%%%%%%%%%%%%%%%%%%%%%%%%%%%%%%%%%%%%%%%%%%%%%%%%%%%%%%%%%%%%%%%%%%%%%%%%%%%%%%%%%%%%%%%%%%%%
		\begin{column}{.39\linewidth}  % задаём ширину первой колонки
			
			\begin{block}{Introduction}
				Here is your introduction. Привет!
			\end{block}
			
			\vspace{2.5cm}  %Небольшой зазор (потом с ним поиграем на конечном этапе)
			
			\begin{block}{Experimental}
				\begin{exampleblock}{Samples preparation}
					The samples were prepared by\ldots
				\end{exampleblock}
				
				\begin{exampleblock}{The structural analysis}
					X-ray powder diffraction (XRPD) was performed at the\ldots
				\end{exampleblock}
				
				\begin{exampleblock}{Oxygen conductivity performance}
					The oxygen conductivity experiments were carried out by\ldots
				\end{exampleblock}
				
				\begin{columns}
					% разобьём данную колонку на два столбца для рисунков
					\column{0.35\textwidth}
					\begin{figure}[h]
						\centering
						%\includegraphics[width=\columnwidth]{membrane-reactor-en}
						\caption{Scheme of the membrane reactor}
						\label{fig:membrane}
					\end{figure}
					
					\column{0.65\textwidth}
					\begin{figure}[h]
						\centering
						%\includegraphics[width=\columnwidth]{all_plot_in_one_1_2_3}
						\caption{Characteristic data obtained
							at the set up}
						\label{fig:oxygen-conductivity}
					\end{figure}
				\end{columns}
				
			\end{block}  %конец Experimental
			
			\begin{exampleblock}{Acknowledgments}
				Many thanks to\ldots
			\end{exampleblock}
		\end{column}  %конец первой колонки
		
		\begin{column}{.59\linewidth}
			\begin{block}{Results and Discussion}
				Although the samples involved were considered to have\ldots
				
				\begin{figure}[h]
					% вставим два рисунка рядом другим образом
					\centering
					%\includegraphics[width=0.49\textwidth]{diffraction_profile}\hfill
					%\includegraphics[width=0.49\textwidth]{lattice-nb02-mm4}\\
					\parbox{0.49\textwidth}{\caption{Diffraction profile of 110 peak}
						\label{fig:diff-profile}}
					\parbox{0.49\textwidth}{\caption{Lattice parameters of the two
							phases}
						\label{fig:lattice-parameters-2phases}}
				\end{figure}
				Fig.~\ref{fig:lattice-parameters-2phases} shows\ldots
				
				\begin{columns}[t]  % разобьём на два столбца.
					% выравнивание столбцов по верхней точке,
					% чтобы не был один выше другого.
					\column{0.48\textwidth}
					\begin{exampleblock}{Phases proportion}
						If one considers the process of structural transformation
						to have\ldots
					\end{exampleblock}
					
					\column{0.48\textwidth}
					\begin{exampleblock}{Oxygen conductivity}
						The thermal dependence of oxygen conductivity through the
						membrane could be written as follows:\ldots
					\end{exampleblock}
				\end{columns}  %конец локального разбиениия
				
				\begin{exampleblock}{The similarity in activation energy values
						and the connection between oxygen conductivity and the
						second phase}
					The $E_{activation}$ values determined from these two methods
					turned out to be rather similar.
				\end{exampleblock}
			\end{block} % конец блока "Results and Discussion"
			
			\begin{block}{Conclusion}
				\begin{itemize}
					\item The first conclusion.
					\item The second conclusion.
					\item The third conclusion.
				\end{itemize}
			\end{block}  %конец "Conclusion"
			
		\end{column}  % конец второй большой колонки постера
		
		%%%%%%%%%%%%%%%%%%%%%%%%%%%%%%%%%%%%%%%%%%%%%%%%%%%%%%%%%%%%%%%%%%%%%%%%%%%%%%%%%%%%%%%%%%%%%%%%%%%%
		%%%%%%%%%%%%%%%%%%%%%%%%%%%%%%%%%%%%%%%%%%%%%%%%%%%%%%%%%%%%%%%%%%%%%%%%%%%%%%%%%%%%%%%%%%%%%%%%%%%%
	\end{columns}  % конец разбиения на колонки
	
	\vfill
	
\end{frame}

\end{document}