%!TEX TS-program = xelatex
% see https://www.writelatex.com/coursera/latex/5.2.2

\documentclass[a4paper,12pt]{extarticle}

% Russian fonts
\usepackage[english,russian]{babel}
\usepackage[utf8]{inputenc}
\usepackage{fontspec}
\defaultfontfeatures{Ligatures={TeX},Renderer=Basic}
\setmainfont[Ligatures={TeX,Historic}]{Times New Roman}
\setmonofont{Courier New}
\usepackage{indentfirst}
\usepackage{parskip}
\frenchspacing

% Margins
% \usepackage[top=20mm, bottom=20mm, left=25mm, right=10mm]{geometry} % GOST
\usepackage[top=20mm, bottom=20mm, left=20mm, right=20mm]{geometry}

% Paragraph spacing
\setlength{\parindent}{2em}
\setlength{\parskip}{1em}

\usepackage{graphicx}

\begin{document} % конец преамбулы, начало документа
	\begin{titlepage}
		\centering
		\textbf{\uppercase{Московский комитет образования\\ГБОУ Лицей № 1533 (информационных технологий)}}\par
		\rule{\textwidth}{1pt}\par
		\vspace{3cm}
		\large\textbf{ВЫПУСКНАЯ РАБОТА}\par
		\vspace{0.5cm}
		\large(специальность "Прикладное программрование")\par
		\vspace{0.5cm}
		\large{учащегося группы 11.3\\Захарова Ильи Александровича}\par
		\vspace{2cm}
		{\Huge{\textbf{Разработка гипервизора Jinet}}}\par
		\vspace{3cm}
		\begin{flushright}
			\begin{tabular}{rl}
				Руководители:& Байков Б.К.\\
							 & Потёмкин А.В.\\
							 \\
				Консультанты:& Завриев Н.К.\\
							 & Гиглавый А.В.
				
			\end{tabular}
		\end{flushright}
		\par
		\vfill
		Москва --- 2017
	\end{titlepage}
	\tableofcontents
	\pagebreak
	\section{Постановка задачи}
	
	Цель настоящей работы – изучение архитектуры Intel x86/x86-64 и написание гипервизора.\par
	Гипервизор – это программа, обеспечивающая разделение ресурсов компьютера на несколько виртуальных машин, и запуск на каждой виртуальной машине своей операционной системы.\par
	Гипервизор обеспечивает либо разделяемый, либо монопольный доступ виртуальных машин к каждому из аппаратных устройств компьютера. Создаются виртуальные устройства, конфигурация которых может отличаться от конфигурации устройств аппаратных.\par
	Гипервизор, в отличие от эмулятора, выполняющего программную эмуляциию команд, лишь перехватывает управление у виртуальных машин в случае необходимости. Код виртуальных машины выполняется аппаратно в процессоре. Современные процессоры Intel подерживают расширения аппаратной виртуализации (VT-i, VT-d), что значительно ускоряет процесс виртуализации.
	
	
	\section{Анализ предметной области}
	Виртуализация – техника предоставления исполняемой программе набора вычислительных ресурсов, абстрагированная от их аппаратной реализации. Виртуализация была предметом изучения информатики на протяжении многих лет: так, например, советские инженеры решали проблему портирования программного обеспечения с платформ имеющих другие интерфейсы, нежели физический компьютер, на котором программа исполнялась.\par
	В рамках этой работы мы будем говорить не столько о виртуализации ресурсов, сколько о работе гипервизора – программы, занимающейся разделением работы ресурсов одной физической машины (т.н. хозяин (\textit{англ.} host)) на множество виртуальных машин (т.н. гость (англ. Guest)). Внутри каждой виртуальной машины исполняется своя ОС, ход работы которой не влияет на работу других ВМ.\par
	Гипервизор, в отличие от эмулятора, выполняющего программную эмуляциию команд, лишь перехватывает управление у виртуальных машин в случае необходимости. Код виртуальных машины выполняется аппаратно в процессоре. Так как принцип работы гипервизора предполагает изоляцию виртуальных машин, для эффективной его работы необходима аппаратная поддержка виртуализации. Первой в этой области была компания IBM с мэйнфреймами System/360, System/370, созданными на рубеже 60-70-х годов прошлого века. Современные процессоры Intel также подерживают расширения аппаратной виртуализации (VT-i, VT-d), что значительно ускоряет процесс виртуализации.\par
	После появления первых гипервизоров появилась необходимость создания формальных критериев виртуализации. В 1974 статья Джеральда Попека и Роберта Гольдберга их сформировала.\par
	\pagebreak
	\subsection{Формальный критерий виртуализуемости}
	Требования к монитору ВМ из состоят из трёх пунктов:
	\begin{enumerate}
		\item \textbf{Изоляция}. Каждая ВМ имеет доступ только к своим ресурсам.
		\item \textbf{Эквивалентность}. Программа, исполняемая под управлением ВМ, демонстрирует поведение, идентичное поведению программы, исполяемой на реальной системе.
		\item \textbf{Эффективность}. Статистически преобладающее подмножество инструкций виртуального процессора должно исполняться напрямую хозяйским процессором, без вмешательства монитора ВМ. Гипервизор перехватывает управление ВМ только в случае исполнения привилегированных операций.
	\end{enumerate}
	\section{Ход}
\bibliography{paper}

\end{document} % конец документа

